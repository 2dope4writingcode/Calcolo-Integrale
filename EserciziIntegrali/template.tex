\documentclass{report}

\input{preamble}
\input{macros}
\input{letterfonts}

\title{\Huge{Calcolo Integrale}\\Risoluzione esercizi}
\author{\huge{David Dragomir}}
\date{1/04/2023}

\begin{document}
\maketitle

\section{Integrazione per sostituzione}

In questa sezione verranno risolti alcuni integrali con
il \textbf{metodo di sostituzione} insieme ad alcuni casi particolari. \\

\qs{}{
$$
\int\limits_0^\pi sin(11x) \, dx
$$
}

\sol{
Sostituendo $y = 11x$, da cui $dy = 11dx$, si ha, sapendo che $cos(11\pi) = -1$:

\begin{align*}
  \int\limits_0^\pi sin(11x) \, dx = \int\limits_0^{11\pi} sin(y)
  \, \frac{dy}{11} = - \frac{cos(y)}{11} \Big|_0^{11\pi} =
  - \frac{cos(11\pi) - 1}{11} = \frac{2}{11}.
  \end{align*}
}

\qs{}{
$$
\int\limits_0^1 e^{4x} \, dx
$$
}

\sol{
Con la sostituzione $y = 4x$, da cui $dy = 4dx$, si ha:

\begin{align*}
  \int\limits_0^1 e^{4x} \, dx = \int\limits_0^4 e^y \, \frac{dy}{4}
  = \frac{e^y}{4} \Big|_0^4 = \frac{e^4- 1}{4}.
\end{align*}
}

\qs{}{
$$
\int\limits_0^1 \frac{10x}{1+5x^2} \, dx
$$
}

\sol{
Con la sostituzione $y = 1+5x^2$, da cui $dy = 10x \, dx$, si ha:

\begin{align*}
  \int\limits_0^1 \frac{10x}{1+5x^2} \, dx = \int\limits_1^6 \frac{dy}{y}
  = ln(|y|) \Big|_1^6 = ln(6) - ln(1) = ln(6).
\end{align*}
}

\qs{}{
$$
\int\limits_0^1 \frac{dx}{1+16x^2} \, dx
$$
}

\sol{}{
Con la sostituzione $y=4x$, da cui $dy = 4dx$, si ha:

\begin{align*}
  \int\limits_0^1 \frac{dx}{1+16x^2} \, dx = \int\limits_0^4 \frac{1}{4}
  \frac{dy}{1+y^2} = \frac{arctan(y)}{4} \Big|_0^4 = \frac{arctan(4)
  - arctan(0)}{4} =   \frac{arctan(4)}{4}.
\end{align*}
}

\qs{}{
$$
\int\limits_0^{\sqrt[3]{\pi}} x^2 cos(7x^3) \, dx
$$
}

\sol{
Con la sostituzione $y = 7x^3$, da cui $dy = 21x^2 dx$, si ha:

\begin{align*}
  \int\limits_0^{\sqrt[3]{\pi}} x^2 cos(7x^3) \, dx = \int\limits_0^{7\pi}
  cos(y) \frac{dy}{21} = \frac{sin(y)}{21} \Big|_0^{7\pi} = 
  \frac{sin(7\pi - sin(0))}{21} = 0.
\end{align*}
}

\qs{}{
$$
\int\limits_0^{1} \frac{dx}{6x+3} \, dx
$$
}

\sol{
Ricordando che:

$$
\int \frac{dy}{ay + b} = \frac{ln(|ay + b|)}{a},
$$
si ha:

\begin{align*}
  \int\limits_0^{1} \frac{dx}{6x+3} \, dx = \frac{ln(|6x+3|)}{6} \Big|_0^1
  = \frac{ln(9) - ln(3)}{6} = \frac{ln(3)}{6}.
\end{align*}
}

\qs{}{
$$
\int\limits_0^{1} \frac{dx}{(3-x)^2} \, dx
$$
}

\sol{
Ricordando che:

$$
\int \frac{dy}{(a-y)^2} = \int \frac{dy}{(y - a)^2} = - \frac{1}{y-a},
$$
si ha:

\begin{align*}
  \int\limits_0^{1} \frac{dx}{(3-x)^2} \, dx = - \frac{1}{x-3} \Big|_0^1
  = \frac{1}{2} - \frac{1}{3} = \frac{1}{6}.
\end{align*}
}

\end{document}
